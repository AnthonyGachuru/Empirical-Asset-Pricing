\documentclass[11pt,reqno,final]{amsart}
\usepackage[font=small,margin=10pt,labelfont={bf},labelsep={space}]{caption}
\usepackage{subfig}
\usepackage{wrapfig}
\usepackage{amsmath, amssymb, epsfig}
\usepackage[scaled]{helvet} % I like Helvetica for sf
\usepackage{fourier}
\usepackage{bm}
\usepackage{color}
\usepackage{fullpage} 	% Fullpage package
%\usepackage{cite}
%\usepackage{citesort}
\newcommand{\notate}[1]{\textcolor{red}{\textbf{[#1]}}}
%\input{macros}


\newcommand{\tasos}{\text{TASOS}}
%Results
%Shortcuts
\newcommand{\hide}[1]{}
\newcommand{\R}{\mathbb{R}}
\newcommand{\E}{\mathbb{E}}
\newcommand{\N}{\mathbb{N}}
\newcommand{\Z}{\mathbb{Z}}

\newcommand{\x}{\mathbf{x}}
\newcommand{\y}{\mathbf{y}}
\newcommand{\z}{\mathbf{z}}
%\newcommand{\A}{\mathbf{A}}
\newcommand{\bi}{\mathbf{b}}
\newcommand{\ro}{\mathbf{r}}
\newcommand{\w}{\mathbf{w}}
\newcommand{\zero}{\mathbf{0}}
\newcommand{\ep}{\epsilon}
\newcommand{\de}{\delta}
\newcommand{\defby}{\overset{\mathrm{\scriptscriptstyle{def}}}{=}}
\newcommand{\bigO}{\mathrm{O}}
\DeclareMathOperator*{\argmin}{arg min}
\DeclareMathOperator*{\argmax}{arg max}
%************************************
%************************************
% The macros below are due to Tassos Zouzias
%************************************
%************************************

\newcommand{\eps}{\varepsilon}


\newcommand{\Prob}[1]{\ensuremath{\mathbb{P}\left(#1\right)}}

\newcommand{\OO}{\mathcal{O}}
\newcommand{\vol}[1]{\text{vol}(#1)}
\newcommand{\tr}{\rm{Tr}}
\newcommand{\RR}{\mathbb{R}}
\newcommand{\NN}{\mathbb{N}}
\newcommand{\reals}{\mathbb{R}}


\newcommand{\e}{\ensuremath{{\rm e}}}
\DeclareMathOperator{\EE}{\mathbb{E}}
% Variance
\newcommand{\var}[1]{\ensuremath{\mathrm{Var}(#1)}}
% Pseudo-inverse of a matrix
\newcommand{\pinv}[1]{ {#1}^\dagger}
\newcommand{\norm}[1]{\ensuremath{\left\|#1\right\|_2}}
\newcommand{\pnorm}[1]{\ensuremath{\left\|#1\right\|_p}}
\newcommand{\qnorm}[1]{\ensuremath{\left\|#1\right\|_q}}
\newcommand{\infnorm}[1]{\ensuremath{\left\|#1\right\|_\infty}}
\newcommand{\onenorm}[1]{\ensuremath{\left\|#1\right\|_1}}
\newcommand{\frobnorm}[1]{\ensuremath{\left\|#1\right\|_{\text{\rm F}}}}
% Stable rank of a matrix
\newcommand{\sr}[1]{\ensuremath{\mathrm{\textbf{\footnotesize sr}}\left(#1\right)}}
% Trace of a matrix.
\newcommand{\trace}[1]{\ensuremath{\mathrm{\textbf{tr}}\left(#1\right)}}
%\DeclareMathOperator{\trace}{trace}
% Rank of a matrix
\newcommand{\rank}[1]{\ensuremath{\mathrm{\textbf{{\footnotesize rank}}}\left(#1\right)}}
% Kernel of a matrix
%\newcommand{\ker}[1]{\ensuremath{\mathrm{\textbf{ker}}\left(#1\right)}}
% Image of a matrix
\newcommand{\im}[1]{\ensuremath{\mathrm{\textbf{Im}}\left(#1\right)}}

% Condition number of a matrix
\newcommand{\cond}[1]{\ensuremath{\mathrm{\text{cond}}\left(#1\right)}}

\newcommand{\expm}[1]{\ensuremath{\mathrm{\textbf{\footnotesize exp}}\left[#1\right]}}
\newcommand{\coshm}[1]{\ensuremath{\mathrm{\textbf{\footnotesize cosh}}\left[#1\right]}}
\newcommand{\detm}[1]{\ensuremath{\mathrm{\textbf{det}}\left(#1\right)}}
\newcommand{\sign}[1]{\ensuremath{\mathrm{\textbf{sign}}\left(#1\right)}}


% # of non-zero entries of a matrix
\newcommand{\nnz}[1]{\ensuremath{\mathrm{\textbf{\footnotesize nnz}}\left(#1\right)}}

% Diagonal Matrix
\newcommand{\diag}[1]{\ensuremath{\mathrm{\textbf{diag}}\left(#1\right)}}
% Polylog(n)
\newcommand{\polylog}[1]{\ensuremath{\mathrm{polylog}\left(#1\right)}}

\newcommand{\old}{\text{old}}
\newcommand{\new}{\text{new}}
\newcommand{\ravg}{\text{R}_{\text{avg}}}
\newcommand{\cavg}{\text{C}_{\text{avg}}}
%\newcommand{\cavg}[1]{\text{C}_{\text{avg}}(#1)}


%%% Vector and matrix operators

\newcommand{\vct}[1]{\bm{#1}}
\newcommand{\mtx}[1]{\bm{#1}}

\newcommand{\ip}[2]{\left\langle {#1},\ {#2} \right\rangle}
\newcommand{\mip}[2]{ {#1}\bullet {#2}}

\newcommand{\ignore}[1]{}

\newcommand{\Id}{\mathbf{I}}
\newcommand{\J}{\mathbf{J}}
\newcommand{\onemtx}{\bm{1}}
%\newcommand{\zeromtx}{\bm{0}}
\newcommand{\zeromtx}{\mathbf{0}}




%%%%%%%%%%%%%%%%%%%%%%%%%%%%%%%%%%%%%
\newcommand{\mat}[1]{ {\ensuremath{\mtx{#1} }}}
%%%%%%%%%%%%%%%%%%%%%%%%%%%%%%%%%%%%%

\def\gammab{{\bm{\gamma}}}
\def\kappab{{\bm{\kappa}}}
\def\sig{{\bm{\Sigma}}}
\def\sigplus{{\bm{\Sigma}^{+}}}
\def\siginv{{\bm{\Sigma}^{-1}}}
\def\bet{{\bm{\beta}}}
\def\one{{\bm{1}}}
\def\exp{\hbox{\rm exp}}
\def\col{\hbox{\rm col}}
\def\ker{\hbox{\rm ker}}
\def\ahat{{\hat\a}}
\def\p{{\mathbf p}}
\def\e{{\mathbf e}}
\def\q{{\mathbf q}}
\def\rb{{\mathbf r}}
\def\s{{\mathbf s}}
\def\u{{\mathbf u}}
\def\v{{\mathbf v}}
\def\d{{\mathbf \delta}}
\def\xhat{{\hat\x}}
\def\yhat{{\hat\y}}
\def\A{\matA}
\def\B{\matB}
\def\C{\matC}
\def\Ahat{\hat\matA}
\def\Atilde{\tilde\matA}
\def\Btilde{\tilde\matB}
\def\Stilde{\tilde\matS}
\def\Utilde{\tilde\matU}
\def\Vtilde{\tilde\matV}
\def\G{{\cl G}}
\def\hset{{\cl H}}
\def\Q{{\bm{Q}}}
\def\U{{\bm{U}}}
\def\V{{\bm{V}}}
\def\win{\hat{\w}}
\def\wopt{\w^*}
\def\matAhat{\hat\mat{A}}
\def\matA{\mat{A}}
\def\matB{\mat{B}}
\def\matC{\mat{C}}
\def\matD{\mat{D}}
\def\matE{\mat{E}}
\def\matH{\mat{H}}
\def\matI{\mat{I}}
\def\matM{\mat{M}}
\def\matP{\mat{P}}
\def\matQ{\mat{Q}}
\def\matR{\mat{R}}
\def\matL{\mat{L}}

\def\matS{\mat{S}}
\def\matT{\mat{T}}
\def\matU{\mat{U}}
\def\matV{\mat{V}}
\def\matW{\mat{W}}
\def\matX{\mat{X}}
\def\matY{\mat{Y}}
\def\matZ{\mat{Z}}
\def\matSig{\mat{\Sigma}}
\def\matOmega{\mat{\Omega}}
\def\matGam{\mat{\Gamma}}
\def\matTheta{\mat{\Theta}}
\def\w{{\mathbf{w}}}
\def\ein{{\cl E_{\text{\rm in}}}}
\def\eout{{\cl E}}
\def\scl{{\textsc{l}}}
\def\scu{{\textsc{u}}}
\def\phiu{{\overline{\phi}}}
\def\psiu{{\overline{\psi}}}
\def\phil{{\underbar{\math{\phi}}}}
\newcommand\remove[1]{}


\newcommand{\vecb}{{\vct{b} }}
\newcommand{\bc}{{\vecb_{\mathcal{R}(\matA)^\bot } }}
\newcommand{\br}{{\vecb_{\mathcal{R}(\matA) } }}

% Least squares solution of Ax = b
\def\xls{\x_{\text{\tiny LS}}}

% For rows and columns of a matrix A
\newcommand{\ar}[1]{ \matA^{(#1)}}
\newcommand{\ac}[1]{ \matA_{(#1)}}

\newcommand{\colspan}[1]{\mathcal{R}(#1)}

\usepackage{palatino}

%---------------------Listings--------------------%
\usepackage{listings}
\usepackage{color}
\definecolor{dkgreen}{rgb}{0,0.6,0}
\definecolor{gray}{rgb}{0.5,0.5,0.5}
\definecolor{mauve}{rgb}{0.58,0,0.82}
\lstset{ %
  language=Octave,                % the language of the code
  basicstyle=\footnotesize,           % the size of the fonts that are used for the code
  numbers=left,                   % where to put the line-numbers
  numberstyle=\tiny\color{gray},  % the style that is used for the line-numbers
  stepnumber=2,                   % the step between two line-numbers. If it's 1, each line
                                  % will be numbered
  numbersep=5pt,                  % how far the line-numbers are from the code
  backgroundcolor=\color{white},      % choose the background color. You must add \usepackage{color}
  showspaces=false,               % show spaces adding particular underscores
  showstringspaces=false,         % underline spaces within strings
  showtabs=false,                 % show tabs within strings adding particular underscores
  frame=single,                   % adds a frame around the code
  rulecolor=\color{black},        % if not set, the frame-color may be changed on line-breaks within not-black text (e.g. commens (green here))
  tabsize=2,                      % sets default tabsize to 2 spaces
  captionpos=b,                   % sets the caption-position to bottom
  breaklines=true,                % sets automatic line breaking
  breakatwhitespace=false,        % sets if automatic breaks should only happen at whitespace
  title=\lstname,                   % show the filename of files included with \lstinputlisting;
                                  % also try caption instead of title
  keywordstyle=\color{blue},          % keyword style
  commentstyle=\color{dkgreen},       % comment style
  stringstyle=\color{mauve},         % string literal style
  escapeinside={\%*}{*)},            % if you want to add LaTeX within your code
  morekeywords={*,...}               % if you want to add more keywords to the set
}
%-------------------------------------------------%

%%%%%%%%%%%%%%%%%%%%%%%%%%%%%%%%%%%%%%%%%%%%%%%%%%%%%%%%%%%%%%%%%%%
%%%%%%%%%%%%%%%%%%%%%%%%%%%%%%%%%%%%%%%%%%%%%%%%%%%%%%%%%%%%%%%%%%%
%%%%%%%%%%%%%%%%%%%%%%%%%%%%%%%%%%%%%%%%%%%%%%%%%%%%%%%%%%%%%%%%%%%
\usepackage{hyperref}

\usepackage[T1]{fontenc}
\usepackage[utf8]{inputenc}
\usepackage{tabularx,ragged2e,booktabs,caption}

% Place this after the backref command
\usepackage{algorithmicx}
\usepackage[ruled]{algorithm}
\usepackage{algpseudocode}

\newcommand{\floor}[1]{\lfloor #1 \rfloor}
\newtheorem{definition}{Definition}
\newtheorem{theorem}{Theorem}
\newtheorem{proposition}[theorem]{Proposition}
\newtheorem{lemma}[theorem]{Lemma}
\newtheorem{corollary}[theorem]{Corollary}
\newtheorem{question}{Question}
\newtheorem{claim}[theorem]{Claim}
\newtheorem{conjecture}[theorem]{Conjecture}
\newtheorem{observation}[theorem]{Observation}
\newtheorem{fact}[theorem]{Fact}
\newtheorem{example}{Example}
%\newtheorem{algorithm}{Algorithm}
\newtheorem{assumption}[theorem]{Assumption}
\newtheorem{remark}{Remark}
\newtheorem{problem}{Problem}

%-------------------------------------------------%


%%%%%%%%%%%%%%
% Document
%%%%%%%%%%%%%%


\title{Parameter Estimation Bias with Different Sample Size}
\author{Ran Zhao}
\thanks{}
\begin{document}

%%%%%%%%%%%%%%%%%%%%%%%%%%%%%%%%%%%%%%%%%%%%%%%%%%
%%%%%%%%%%%%%%%%%%%%%%%%%%%%%%%%%%%%%%%%%%%%%%%%%%
\begin{abstract}
This paper investigates the bias of parameter estimation, given different sets of parameter true value and sample size. We found that non-zero $\alpha$ improves the unbiasedness of parameter estimation in AR(1) model, but the level of $\sigma$ does not contribution to unbiasedness. Another improvement of parameter estimation raises when the AR(1) becomes stationary, say changing the $\beta$ from 1 to 0.95. Increasing the sample size makes the distribution of $\beta$'s t-statistic closer to be normally distributed. 
\end{abstract}

\maketitle
%%%%%%%%%%%%%%%%%%%%%%%%%%%%%%%%%%%%%%%%%%%%%%%%%%%
%%%%%%%%%%%%%%%%%%%%%%%%%%%%%%%%%%%%%%%%%%%%%%%%%%%
%
%
%
\section{Model Specification}
Consider the AR(1) model with dynamic
\begin{equation}\label{ar}
p_t = \alpha +  \beta p_{t-1} + \epsilon_t, \quad \epsilon_t \sim N(0, \sigma^2)
\end{equation}
where the error terms, $\epsilon_t$, are i.i.d.

Given the true value of the parameters and the initial value of the log stock price, the full log stock price process, $\{p_t\}_{t=0}^{T}$, can be simulated by the following algorithm,

\makeatletter
\def\BState{\State\hskip-\ALG@thistlm}
\makeatother

\begin{algorithm}
\caption{Estimate Bias}\label{algo}
\begin{algorithmic}[1]
\Procedure{Simulation}{}
\BState set sample size $N$ and simulation length $T$
\BState \emph{loop i from 1 to N}:
\State initialize $p(0,i)$
\State \emph{loop t from 1 to T}:
\State \quad draw random normal variable $u\sim N(0, \sigma^2)$
\State \quad $p(t,i) \gets \alpha+\beta p(t,i) + norm $
\State next $t$
\State estimate parameter bias $Bias_{(\theta,i)}[\hat{\theta}]=\hat{\theta}_i-\theta$, for $\theta=\{\alpha, \beta, \sigma\}$
\BState next $i$
\BState $Bias_{(\theta)}[\hat{\theta}]=\sum_{i}Bias_{(\theta,i)}/N$
\EndProcedure
\end{algorithmic}
\end{algorithm}

\section{Estimation Analysis}
\subsection{$\alpha=0, \beta=1, \sigma=0.2$}
First, we select sample size $T=50$ with 10,000 paths to estimate the bias of parameters estimation. Given the parameter set as $\theta=\{\alpha=0, \beta=1, \sigma=0.2\}$, the biases are shown in Table~\ref{tbl::sim_results}. The initial value of the log stock price is set to $\log(100)\approx 4.605$. The bias for $\alpha$ estimation is 0.4739, the bias for $\beta$ estimation is -0.1029, and the bias for $\sigma$ is -0.0049. 

When bumping the true value of $\alpha$, the bias of all the parameters reduce significantly. That is, a non-zero trending benefits the unbiasedness of the parameter estimation. However, bumping $\sigma$ up and down does not improve the bias of parameter estimation. 

To test $\beta=1$, the t-statistic is calculated as $t=(\hat{\beta}-1)/se(\hat{\beta})$. With the sample size of 50, the 1\% and 5\% t-statistics are -3.5265 and -2.9083, respectively. 

When sample size increases to 600, the 1\% and 5\% t-statistics are -2.6658 and -1.9607, respectively. With the increase of sample size, the t-statistics are closer to normal distributed statistics, with narrower tail-distribution. When sample size is smaller, the fat-fail is more obvious. 

\begin{table}
\begin{center}
\caption{The bias on parameter estimation and t-statistic on $\beta$ estimation are provided, given different sets of true parameter values. The model specification is as Equation~\ref{ar}. Simulation procedure follows Algorithm~\ref{algo}. }
\begin{tabular}{|c|c|c|c|c|c|c|c|c|c|}
  \hline
Run Tag	&	$\alpha$	&	$\beta$	&	$\sigma$	&	$T$	&	$Bias(\hat{\alpha})$	&	$Bias(\hat{\beta})$	&	$Bias(\hat{\sigma})$	&	$t_{\beta}(1\%)$	&	$t_{\beta}(5\%)$	\\ \hline
1	&	0	&	1	&	0.2	&	50	&	0.4739	&	-0.1029	&	-0.0049	&	-3.5265	&	-2.9083	\\
2	&	-0.2	&	1	&	0.2	&	50	&	-0.0049	&	-0.0026	&	-0.0012	&	-2.6658	&	-1.9607	\\
3	&	0.2	&	1	&	0.2	&	50	&	0.0266	&	-0.0024	&	-0.0015	&	-2.5619	&	-1.8682	\\
4	&	0	&	1	&	0.1	&	50	&	0.4754	&	-0.1033	&	-0.0025	&	-3.5554	&	-2.9260	\\
5	&	0	&	1	&	0.3	&	50	&	0.4860	&	-0.1051	&	-0.0078	&	-3.6126	&	-2.9336	\\
6	&	0	&	1	&	0.2	&	600	&	0.0413	&	-0.0090	&	-0.0004	&	-3.5049	&	-2.8769	\\
7	&	0	&	0.95	&	0.2	&	50	&	0.0183	&	-0.0126	&	-0.0011	&	-4.7079	&	-3.9881	\\
  \hline
\end{tabular}\label{tbl::sim_results}
\end{center}
\end{table}


\subsection{$\alpha=0, \beta=0.95, \sigma=0.2$}
Changing true value of $\beta$ to be 0.95 instead of 1, or $\alpha$ estimation is 0.0183, the bias for $\beta$ estimation is -0.0126, and the bias for $\sigma$ is -0.0011. The biases when $\beta=0.95$ are consistently lower than the biases when $\beta=1$, showing an improving on unbiasedness when the model is stationary. 

%
%
%%%%%%%%%%%%%%%%%%%%%%%%%%%%%%%%%%%%%%%%%%%%%%%%%%%
%\newpage
%\bibliographystyle{plain}
%\bibliography{bib}
%%%%%%%%%%%%%%%%%%%%%%%%%%%%%%%%%%%%%%%%%%%%%%%%%%%
%%%%%%%%%%%%%%%%%%%%%%%%%%%%%%%%%%%%%%%%%%%%%%%%%%%
%%%%%%%%%%%%%%%%%%%%%%%%%%%%%%%%%%%%%%%%%%%%%%%%%%%
%
%
\newpage
\section*{Code Appendix}
%\begin{spacing}{0.9}
\lstinputlisting[language=R]{assignment3.R}

\end{document}
\endinput
